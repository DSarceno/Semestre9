\vspace{0.5cm}

Las enfermedades desmielinizantes pueden heredarse en niños o ser causada por diferentes razones en los adultos. En los niños se tienen enfermedades hereditarias que hacen que las vainas de mielina no se desarrollen bien, algunas de estas son: enfermedad de Tay-Sachs, de Niemann-Pick o el síndrome de Hurler. Para los adultos se tienen diferentes causas para el daño/destrucción de las vainas de mielina: Algún ictus, infecciones, trastornos inmunitarios o metabólicos, falta de vitamina B12, consumo excesivo de alcohol :(, etc. Algunos trastornos desmielinizantes se describen brevemente a continuación.

\section*{Esclerosis Múltiple}
Esta es una enfermedad el grave daño a los nervios interrumpe la comunicación entre el cerebro en el cuerpo, ocacionando síntomas tales como: pérdida de visión, dolor, fatiga y disminución de la coordinación. La progreción, gravedad y duración depende de la persona.

\section*{Encefalomielitis Aguda Diseminada}
Este desorden ataca el sistema nervioso central con inflamación. Los síntomas son: fiebre alta, dolor de cabeza, nauseas, vomitos, confusión, convulciones y hasta coma.


\section*{Trastorno del Espectro de la Neuromielitis Óptica}
Así como la encefalomielitis aguda diseminada, este trastorno ataca el sistema nervioso central e invade los nervios ópticos, causando pérdida de visión y parálisis.


