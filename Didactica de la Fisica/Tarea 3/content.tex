\vspace{0.5cm}

\begin{description}
	\item[Estímulo: ] Es un estímulo externo por medio del oído e incluso la vista (al ver el contacto de quién llama).
	\item[Registro Sensorial:] Es el almacenar la pregunta o propuesta a salir a comer sushi.
	\item[Atención: ] Centrarse en la conversación como tal y no en estimulos externos irrelevantes en la misma.
	\item[Memoria Operativa: ] Mientras se piensa en la respuesta a la propuesta, se guarda y mantiene en conciencia.
	\item[Percepción: ] 
	\item[Codificación: ] Se introduce a la memoria la propuesta y el hecho de que la conversación esté ocurriendo.
	\item[Emoción: ] La emoción, valga la redundancia, de su amigo invitandolo a comer y la propia, o no, al ser invitado.
	\item[Recuperación: ] Se accede a la memoria a largo plazo sobre información de enventos futuros (agenda), para ver si está o no ocupado.
	\item[Memoria a Largo Plazo: ] En caso de que no, se almacena el nuevo evento para una futura recuperación.
	\item[Procesos ejecutivos: ] Luego de todo el \textbf{razonamiento} realizado se procede a dar respuesta por medio verbal, utilizando la \textbf{condición motora} y \textbf{lenguaje} para dar a entender de forma correcta nuestra desición.
\end{description}











%%%%
