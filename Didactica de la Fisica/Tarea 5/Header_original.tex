
% AUTOR: Diego Sarceño

% ENCABEZADO DE TRABAJOS CON LOGO DE LA UNIDAD ACADÉMICA

% ENCABEZADO LOGO COLOR
%\begin{tabulary}{20cm}{Lp{0.9cm}p{6.1cm}}
%Universidad de San Carlos de Guatemala & & \multirow{4}{8cm}{\hfill %\includegraphics[scale=0.5]{ECFM.png}}\\            % Logo de la unidad academica
%Escuela de Ciencias Físicas y Matemáticas & \hfill & \\
%Diego Sarceño 201900109 & \hfill & \\
%Análisis de Variable Compleja 1 & \hfill & \\
%\today & & \\	
%\end{tabulary}\\[0.25cm]


% ENCABEZADO LOGOS
%\begin{tabulary}{20cm}{LLCRR}
%\multirow{4}{2.3cm}{\includegraphics[scale=0.13]{/home/diego/Documents/Licenciatura/LatexBasic/ECFM.pdf}} & Universidad de San Carlos de Guatemala  & & & \multirow{4}{4.5cm}{\hfill \includegraphics[scale=0.082]{/home/diego/Documents/Licenciatura/LatexBasic/USAC.pdf}}\tabularnewline
% & Escuela de Ciencias Físicas y Matemáticas & \hfill &  & \tabularnewline
% & Física Computacional & \hfill ~~ &   & \tabularnewline
% & Diego Sarceño 201900109 & &  & \tabularnewline
% & \today &  & & \tabularnewline
%\end{tabulary}\\[0.75cm]
%
%{\hrule height 1.5pt} \vspace{0.1cm}
%\begin{tabulary}{21cm}{p{5.5cm}p{11cm}p{3.8cm}}
%    \hfill & \huge{\scshape{Tarea 1}} & \hfill
%\end{tabulary}
%{\hrule height 1.5pt} 
%\vspace{0.5cm}


\textcolor{DS_Black}{
\begin{minipage}{0.85\textwidth}
    \begin{center}
        \textbf{\Large Tarea 2}\\
        \vspace{5pt}
        Didáctica de la Física \\
        \vspace{20pt}
        \textit{Diego Sarceño} \\
        \vspace{5pt}
        \footnotesize{\textit{201900109}} \\
        \vspace{5pt}
        \today
    \end{center}
\end{minipage}
\vspace{10pt}
\hrule
}