\vspace{0.5cm}

\begin{multicols}{2}


\section*{Serotonina}
\subsection*{Función}
La serotonina es un neurotransmisor y vasoconstrictor\footnote{Sustancia que hace que los vasos sanguineos se estrechen.} el cual tiene la función de regular el sueño, los estados de ánimo y tiene influencia sobre el funcionamiento vascular y la frecuencia del latido del corazón. Al influir sobre los estados de ánimo, la serotonina, funciona como modulador e inhibidor de la conducta. También regula la timia, que es el comportamiento exterior de individuo, sueño, actividad sexual, apetito, funciones neuroendocrinas, dolor, etc.
\subsection*{Causas de que se altere}
Algunos de los factores que pueden afectar la serotonina, son:
\begin{itemize}
	\item Estrés.
	\item Manejo disfuncional de situaciones.
	\item Cambios hormonales.
	\item Diabetes.
	\item Deficiencia de L-triptófano y vitamina B6.
	\item Medicamentos con serotonina, como medicamentos para la migraña o antidepresivos.
\end{itemize}
\subsection*{Consecuencias de que se altere}
La alteración de la serotonina afecta directamente sus funciones, lo que desemboca en los siguientes signos y síntomas:
\begin{itemize}
	\item Agitación.
	\item Confusión.
	\item Presión alta.
	\item Pérdida de coordinación, espasmos musculares o convulsiones.
	\item Diarrea.
	\item Dolor de cabeza.
	\item Fiebre alta.
	\item Pérdida del conocimiento.
\end{itemize} 



\section*{Dopamina}
\subsection*{Función}
Sin ahondar mucho en cada una de las distintas funciones, la dopamina afecta en el sistema nervioso la anatomía, el movimiento, cognición y corteza frontal, solcialización, regula la prolactina, entre otros. También tiene funciones en el comportamiento como el humor, la atención y el aprendizaje.
\subsection*{Causas de que se altere}
\begin{itemize}
	\item La dopamina es culpable de las adicciones, así que la droga es uno de los principales causantes, así como cualquier sustancia/cosa que nos genere placer, aumenta la cantidad de dopamina.
	\item Dieta con contenido alto en tirosina.
\end{itemize}
\subsection*{Consecuencias de que se altere}
Para exceso o falta de dopamina, se tienen las siguientes consecuencias
\begin{itemize}
	\item Esquizofrenia.
	\item Trastorno bipolar. 
	\item Déficit de atención.
	\item Enfermedad de Huntington.
	\item Parkinson.
	\item Neurodejeneración asociada al VIH-1.
	\item Depresión.
\end{itemize}




\section*{Endorfina}
\subsection*{Función}
Las endorfinas sirven en el cuerpo como analgésico, producen placer, sensación de bienestar, colaboran al aumentar el orgasmo, en situaciones de hambre y "crean felicidad".
\subsection*{Causas de que se altere}
En forma sana, las endorfinas se pueden variar mediante la actividad física, meditación o, incluso, la risa.
\subsection*{Consecuencias de que se altere}
El déficit de endorfinas puede provocar estados de depresión o desequilibrio emocional. Un exceso puede llegar a causar hiperactividad lo que elevaría el ánimo de la persona a tal grado de no ser capaz de identificar lo agradable de lo desagradable bloqueando la respuesta al dolor, cosa que es muy poco probable pero es una posibilidad.



\section*{Adrenalina}
\subsection*{Función}
Una de las funciones más primitivas y conocidas de la adrenalina es que nos permite reaccionar rápidamente en situaciones de peligro; además, puede ser utilizada en situaciones de paro cardiaco, anafilaxia, larigitis o asma grave.
\subsection*{Causas de que se altere}
El ejercicio, alguna situación de peligro son causas del aumento de la adrenalina. La insuficiencia adrenal es causada por daño en las glándulas suprarrenales, daño en la hipófisis, deshidratación o alguna infección.
\subsection*{Consecuencias de que se altere}
\begin{itemize}
	\item Dolor abdominal.
	\item Confusión o pérdida del conocimiento y, en casos extremos, coma.
	\item Deshidratación.
	\item Vértigo.
	\item Fatiga, dolor de cabeza o fiebre alta.
	\item Presión baja (para insuficiencia), presión alta para exceso.
	\item Alta frecuencia respiratoria.
\end{itemize}




\section*{Noradrenalina}
\subsection*{Función}
La función principal de la noradrenalina es disminuir la presión diastólica\footnote{La presión diastólica es aquella que se presenta cuando el corazón se relaja y se llena de sangre.}.
\subsection*{Causas de que se altere}
Así como en la adrenalina, la noradrenalina se ve afectada por alteración o malfuncionamiento de las glandulas suprarrenales. Además de otras causas como el estrés crónico.
\subsection*{Consecuencias de que se altere}
La falta de noradrenalina puede provocar:
\begin{itemize}
	\item Hipotensión arterial.
	\item Bradicardia.
	\item Hipotermia.
	\item Depresión. (Otra vez)
\end{itemize}
Mientras que el exceso de noradrenalina puede provocar:
\begin{itemize}
	\item Taquicardia.
	\item Hipertensión.
	\item Dolor torácico.
	\item Falta de flujo sanguíneo a los órganos vitales.
\end{itemize}



\section*{Glutamato}
\subsection*{Función}
El glutamato regula los sistemas motores, sensitivos y cognitivos, también tiene relevancia en la plasticidad sináptica.
\subsection*{Causas de que se altere}
El incremento del calcio neuronal causa variaciones de glutamato.
\subsection*{Consecuencias de que se altere}
Lás alteraciones de glutamato dañan el sistema nervioso, sobre-estimulan las neuronas llevandolas a un estado de agotamiento e incluso muerte, además de llevar a malestares físicos.




\section*{GABA}
\subsection*{Función}
El ácido gamma-aminobutírico se utiliza para cumplir funciones específicas en la interrupción de la transmisión de los impulsos nerviosos entre neuronas y en el ámbito psicológico tiene relevancia en la ansiedad y depresión. También ayuda en el fortalecimiento de la memoria.
\subsection*{Causas de que se altere}
Puede variar por diversas bacterios acidolácticas durante la fermentación.
\subsection*{Consecuencias de que se altere}
El aumento de GABA aumenta las ondas cerebrales asociadas a un estado relajado. Y el desequilibrio en sus concentraciones puede desembocar en trastornos como el autismo.



\section*{Acetilcolina}
\subsection*{Función}
La función de la acetilcolina es contraer la musculatura lisa, dilatar los vasos sanguíneos y disminuir la frecuencia cardíaca.
\subsection*{Causas de que se altere}
Una de las causas del aumento o disminución de la acetilcolina es la disminución o aumento de la dopamina.
\subsection*{Consecuencias de que se altere}
Dada las funciones de este neurotransmisor, las consecuencias de su alteración en el cuerpo puede recaer en enfermedades como el Parkinson y Alzheimer.






\end{multicols}











%%%%
