\documentclass[11pt]{beamer}
\usetheme{Warsaw}
\usepackage[utf8]{inputenc}
\usepackage[spanish]{babel}
\usepackage{amsmath,amsthm,amssymb} %modos matemáticos y  simbolos
\usepackage{latexsym,amsfonts} %simbolos matematicos
\usepackage{graphicx}
\usepackage{physics} %Simbolos fisicos
\usepackage{array} %mejores formatos de tabla
\usepackage{tabulary}
\usepackage{multirow} %ocupar varias filas en una tabla
\usepackage{fancybox} %recuadros talegas
\usepackage{float} %ubicar graficas
\usepackage{color}
\usepackage{comment}
\usepackage{tikz}
\usepackage{stackrel}
\usepackage{calligra}
\usepackage{lipsum} % texto de relleno
\usepackage{cite}
\author{Diego Sarceño \\ \footnotesize{201900109}}
\title{Osciladores Acoplados \\ \footnotesize{Didáctica de la Física}}
%\setbeamercovered{transparent} 
%\setbeamertemplate{navigation symbols}{} 
%\logo{} 
%\institute{} 
\date{\today} 
%\subject{} 
\begin{document}

\begin{frame}
\titlepage
\end{frame}

%\begin{frame}
%\tableofcontents
%\end{frame}

\frame{
	\frametitle{Capítulo 9 Problema 6 Enunciado}
	MAMABICHO
}


\frame{
	


\tikzset{every picture/.style={line width=0.75pt}} %set default line width to 0.75pt        

\begin{tikzpicture}[x=0.75pt,y=0.75pt,yscale=-1,xscale=1]
%uncomment if require: \path (0,307); %set diagram left start at 0, and has height of 307

%Shape: Inductor (Air Core) [id:dp6076043388715762] 
\draw   (264.21,109) -- (264.21,123.4) .. controls (274.72,123.61) and (283.7,125.61) .. (286.85,128.43) .. controls (289.99,131.26) and (286.65,134.35) .. (278.43,136.2) .. controls (272.02,137.63) and (263.73,138.21) .. (255.69,137.8) .. controls (252.55,137.8) and (250,137.08) .. (250,136.2) .. controls (250,135.32) and (252.55,134.6) .. (255.69,134.6) .. controls (263.73,134.19) and (272.02,134.77) .. (278.43,136.2) .. controls (285.26,137.86) and (289.13,140.18) .. (289.13,142.6) .. controls (289.13,145.02) and (285.26,147.34) .. (278.43,149) .. controls (272.02,150.43) and (263.73,151.01) .. (255.69,150.6) .. controls (252.55,150.6) and (250,149.88) .. (250,149) .. controls (250,148.12) and (252.55,147.4) .. (255.69,147.4) .. controls (263.73,146.99) and (272.02,147.57) .. (278.43,149) .. controls (285.26,150.66) and (289.13,152.98) .. (289.13,155.4) .. controls (289.13,157.82) and (285.26,160.14) .. (278.43,161.8) .. controls (272.02,163.23) and (263.73,163.81) .. (255.69,163.4) .. controls (252.55,163.4) and (250,162.68) .. (250,161.8) .. controls (250,160.92) and (252.55,160.2) .. (255.69,160.2) .. controls (263.73,159.79) and (272.02,160.37) .. (278.43,161.8) .. controls (286.65,163.65) and (289.99,166.74) .. (286.85,169.57) .. controls (283.7,172.39) and (274.72,174.39) .. (264.21,174.6) -- (264.21,189) ;
%Shape: Inductor (Air Core) [id:dp6332684177574848] 
\draw   (326.79,189) -- (326.79,174.6) .. controls (316.28,174.39) and (307.3,172.39) .. (304.15,169.57) .. controls (301.01,166.74) and (304.35,163.65) .. (312.57,161.8) .. controls (318.98,160.37) and (327.27,159.79) .. (335.31,160.2) .. controls (338.45,160.2) and (341,160.92) .. (341,161.8) .. controls (341,162.68) and (338.45,163.4) .. (335.31,163.4) .. controls (327.27,163.81) and (318.98,163.23) .. (312.57,161.8) .. controls (305.74,160.14) and (301.87,157.82) .. (301.87,155.4) .. controls (301.87,152.98) and (305.74,150.66) .. (312.57,149) .. controls (318.98,147.57) and (327.27,146.99) .. (335.31,147.4) .. controls (338.45,147.4) and (341,148.12) .. (341,149) .. controls (341,149.88) and (338.45,150.6) .. (335.31,150.6) .. controls (327.27,151.01) and (318.98,150.43) .. (312.57,149) .. controls (305.74,147.34) and (301.87,145.02) .. (301.87,142.6) .. controls (301.87,140.18) and (305.74,137.86) .. (312.57,136.2) .. controls (318.98,134.77) and (327.27,134.19) .. (335.31,134.6) .. controls (338.45,134.6) and (341,135.32) .. (341,136.2) .. controls (341,137.08) and (338.45,137.8) .. (335.31,137.8) .. controls (327.27,138.21) and (318.98,137.63) .. (312.57,136.2) .. controls (304.35,134.35) and (301.01,131.26) .. (304.15,128.43) .. controls (307.3,125.61) and (316.28,123.61) .. (326.79,123.4) -- (326.79,109) ;
%Shape: Capacitor [id:dp4648481053893567] 
\draw   (406,108) -- (406,144) (426,152) -- (386,152) (426,144) -- (386,144) (406,152) -- (406,188) ;
%Shape: Capacitor [id:dp021548038290668314] 
\draw   (186.77,109) -- (186.98,145) (207.02,152.88) -- (167.02,153.12) (206.98,144.88) -- (166.98,145.12) (187.02,153) -- (187.23,189) ;
%Straight Lines [id:da6200762765039554] 
\draw    (186.77,109) -- (264.21,109) ;
%Straight Lines [id:da4677723593786227] 
\draw    (187.23,189) -- (264.21,189) ;
%Straight Lines [id:da42083915396423954] 
\draw    (326.79,109) -- (406,108) ;
%Straight Lines [id:da19414008963857543] 
\draw    (326.79,189) -- (406,188) ;




\end{tikzpicture}

}

\frame{
	$$ q_1 (t) = X_1 \cos{\frac{(\omega _1 + \omega _2)t}{2}} \cos{\frac{(\omega _1 - \omega _2)t}{2}} + X_2 \sin{\frac{(\omega _1 + \omega _2)t}{2}} \sin{\frac{(\omega _2 - \omega _1)t}{2}} $$
	
	$$ q_2 (t) = X_1 \sin{\frac{(\omega _1 + \omega _2)t}{2}} \sin{\frac{(\omega _2 - \omega _1)t}{2}} + X_2 \cos{\frac{(\omega _1 + \omega _2)t}{2}} \cos{\frac{(\omega _1 - \omega _2)t}{2}} $$
}


\frame{
	\centering
	\vspace{1cm}
	GRACIAS POR SU ATENCIÓN $<3$
}














\end{document}