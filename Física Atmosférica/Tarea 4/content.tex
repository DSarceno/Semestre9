\vspace{0.5cm}


\section*{Problema 2}




	
\section*{Problema 7}
Para una atmósfera isotérmica se tiene que la presión varía de la siguiente forma
	$$ p(z) = p_s e^{-\frac{z}{H}}, $$
con $H = 8.192km$. El potencial de temperatura es
	$$ \theta = T\qty(\frac{p_s}{p})^\kappa, \quad \kappa = 2/7. $$
Valuando para cada altitud:
	\begin{align*}
		p(5km) = 543hPa \quad \Rightarrow \quad \theta (5km) = 280*(1000/543)^{2/7} &= 333K \\
		p(10km) = 295hPa \quad \Rightarrow \quad \theta (10km) = 280*(1000/295)^{2/7} &= 397K \\
		p(20km) = 87hPa \quad \Rightarrow \quad \theta (20km) = 280*(1000/87)^{2/7} &= 563K
	\end{align*}
El potencial de temperatura será el mismo que cuando salío, por lo que la temperatura será
	$$ T = \theta \qty(\frac{p}{p_s})^\kappa = 397*\qty(\frac{543}{1000})^{2/7} = 333K. $$



















%%%%%