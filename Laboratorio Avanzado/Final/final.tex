\documentclass[conference]{IEEEtran}
\IEEEoverridecommandlockouts
% The preceding line is only needed to identify funding in the first footnote. If that is unneeded, please comment it out.
\usepackage{amsmath,amsthm,amssymb} %modos matemáticos y  simbolos
\usepackage{latexsym,amsfonts} %simbolos matematicos
\usepackage{cancel} %hacer la linea que cancela las ecuaciones
\usepackage[spanish, es-noshorthands]{babel} %comandos en español y cambia el cuadro por la tabla
\decimalpoint %cambia las comas por puntos decimal
\usepackage[utf8]{inputenc} %caracteristicas del español
\usepackage{physics} %Simbolos fisicos
\usepackage{array} %mejores formatos de tabla
\parindent =0cm %sangria 
\usepackage{algorithmic}
\usepackage{graphicx}
\usepackage{textcomp}
\usepackage{xcolor}
\usepackage{mathtools} 
\usepackage[framemethod=TikZ]{mdframed}%Entornos talegas
\usepackage[colorlinks = true,
			linkcolor = blue,
			citecolor = black,
			urlcolor = blue]{hyperref}%formato de los links y URL's
\usepackage{multicol} %varias columnas
\usepackage{enumerate} %enumeraciones
\usepackage{pgf,tikz,pgfplots} %documentos en formato tikz
\usepackage{mathrsfs} %letras chingonas (transformada de laplace)
\usepackage{subfigure} %varias figuras seguidas
\usepackage{tabulary}
\usepackage{multirow} %ocupar varias filas en una tabla
\usepackage{fancybox} %recuadros talegas
\usepackage{float} %ubicar graficas
\usepackage{color}
\usepackage{comment}
\usepackage{stackrel}
\usepackage{calligra}
\usepackage{lipsum}
\usepackage{cite}
\pgfplotsset{compat=1.16} 

\newcommand{\R}{\mathbb{R}}
\newcommand{\Z}{\mathbb{Z}}
%%%%%%%%%%%%%%%%%%%%%%%%%%%%%%%%%%%%%%%%%%%%%%%%%%%%%%
\def\BibTeX{{\rm B\kern-.05em{\sc i\kern-.025em b}\kern-.08em
    T\kern-.1667em\lower.7ex\hbox{E}\kern-.125emX}}
\begin{document}

\title{Eficiencia de un Detector Geiger-Müller \\
{\footnotesize \scshape{Proyecto Final}}
}

\author{\IEEEauthorblockN{1\textsuperscript{st} Diego Sarceño Ramírez}
\IEEEauthorblockA{\textit{201900109} 
}
%\and
%\IEEEauthorblockN{2\textsuperscript{nd} Andrés Pérez}
%\IEEEauthorblockA{\textit{201704199}
%}
%\and
%\IEEEauthorblockN{3\textsuperscript{rd} Diego Sarceño Ramírez}
%\IEEEauthorblockA{\textit{201900109} \\
%}
}



\maketitle

\begin{abstract}

\end{abstract}

\begin{IEEEkeywords}
	Detector Geiger-Müller, contador Geiger, radiaciones ionizantes, radiactividad.
\end{IEEEkeywords}

%\section{Objetivos}
%
%\subsection{General}
%    \begin{enumerate}[1.]
%        \item Realizar un circuito sumador/restador de 3 bits de entrada con una salida de resultado en un display de 7 segmentos en formato base 10.
%    \end{enumerate}
%\subsection{Específicos}
%    \begin{enumerate}
%        \item Diseñar múltiples circuitos combinacionales para obtener un resultado único en conjunto.
%        \item Implementar un circuito de lógica combinacional capaz de realizar operaciones aritméticas simples utilizando únicamente compuertas lógicas.
%        \item Optimizar el uso de compuertas mediante técnicas distintas al uso de Mapas de Karnaugh.
%        \item Contrastar los diseños teóricos con los resultados experimentales de los circuitos implementados físicamente.
%    \end{enumerate}
%\section{Introducción}
    
\section{Marco Teórico}





%\section{Diseño Experimental}



\section{Montaje Experimental}
%    \subsection{Materiales a Utilizar}
%        \begin{itemize}
%    	\item 
%    \end{itemize}
%
%    \subsection{Procedimientos}
%        \begin{enumerate}
%            \item 
%        \end{enumerate}


        
        
        
        
\section{Resultados}

    
    
    
\section{Discusión de Resultados}
\begin{enumerate}
    \item 
    \item 
    \item 
\end{enumerate}



\section{Conclusiones}
\begin{enumerate}
    \item 
    \item 
\end{enumerate}






\section{Recomendaciones}







\section{Anexos}

        
        
        
\begin{thebibliography}{00}
\bibitem{b2} \textit{Chapter: Histograms}. \url{https://root.cern.ch/root/htmldoc/guides/users-guide/Histograms.html}
\end{thebibliography}

\end{document}


